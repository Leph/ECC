\documentclass[a4paper, 11pt]{article}
\usepackage{listings}
\usepackage[utf8]{inputenc}
\usepackage{color}
\usepackage{graphicx}
\usepackage{multicol}
\usepackage{verbatim}
\usepackage{lmodern}
\usepackage[french]{babel}
\usepackage{listings}
\usepackage[T1]{fontenc}

\title{Projet de Compilation, Backend}
\author{Lun Juang, Quentin Rouxel}
\date{23/01/2012}

\begin{document}
\maketitle{}

\section{Introduction}

L'objectif de ce projet était la réalisation d'un petit compilateur basé sur un langage proche du C et orienté vectoriel.
Nous nous sommes attachés à l'écriture du backend de ce compilateur. A partir d'un langage intermédiaire dont la syntaxe et la sémantique
a préalablement été vérifiées par le frontend, il s'agissait de générer le code assembleur correspondant et à destination d'une architecture
x86, 32 bits.

\section{Langage intermédiaire}

\subsection{Réécriture de la grammaire}

Il s'est avéré que le fichier yacc implémentant la grammaire du langage intermédiaire qui était fournie comportait quelques erreurs.
Plus exactement, la grammaire acceptait des codes qui ne faisait clairement pas partie ni du langage d'entrée, ni du langage intermédiaire.
Pour ne pas avoir a gérer des erreurs syntaxiques, nous avons choisi de réécrire en grande partie la grammaire 
pour d'une part, la rendre plus strict, plus cohérente avec la langage intermédiaire utilisé ici et d'autre part un peu plus simple à
utilisée dans le cadre de l'arbre syntaxique.

\subsection{Utilisation de Lex et Yacc}

Nous nous somme basé sur le document suivant pour comprendre le fonctionnement détaillé de Lex et de Yacc. 
http://www.scribd.com/doc/8669780/Lex-yacc-Tutorial.
L'instruction union de yacc permet de spécifier le type des valeurs contenue dans l'arbre syntaxique. Ensuite, la déclaration
\%token<nom-type> (respectivement \%type<nom-type>) permet de définir le champ à utiliser dans le union pour atteindre la valeur des tokens, 
respectivement des règles de la grammaire.

\subsection{Reconstruction de l'arbre syntaxique}

Au lieu de générer le code assembleur directement dans les règles sémantiques de Yacc, nous avons préféré nous abstraire du parcours 
postfixe assez contraignant de l'arbre syntaxique. Ce parcours étant imposé par l'algorithme réalisant l'analyse syntaxique. Nous avons
pour ce faire recopié la structure complète de arbre syntaxique afin de la manipuler plus simplement par la suite. Ce choix s'explique 
également par la nécessité de pouvoir manipuler le code si nous voulions faire des optimisations dessus.
Le fichier ecc.h et ecc.c contiennent respectivement la définition et l'implémentation de cette structure de l'arbre syntaxique.
Les règles sémantiques associés à chaque règle de la grammaire ont alors pour rôle de construire la représentation du code et de le faire
remonter par l'arbre syntaxique.\\

Notons ici que l'utilisation d'un langage orienté objet aurai certainement simplifié cette construction fastidieuse.\\

Deux opérations importantes sont effectué après la construction de la structures de données et avant la génération du code : 
le cha\^inage des tables de variables et le calcul des offsets pour chaque variable.
Il s'agit pour le premier de lier la structure qui nous sert de table de symbole à sa table englobante.
Pour le deuxième, on calcul par rapport à \%epb, le décalage à effectuer sur la pile pour accéder à la variable donnée. Positif 
pour les arguments d'une fonction, négatif pour les variables locales.\\

Remarque : Valgrind nous assure que notre compilateur ne génère pas de fuite de mémoire dans notre gestion dynamique des structures.

\section{Génération du code assembleur}

\subsection{Protocole d'appel}

Notre protocole d'appel des fonctions est le protocole standard sous Linux :\\
Appelant :\\
1 : sauvegarde des registres utilisés\\
2 : empilement des paramètres sur la pile\\
3 : appel à call (empilement de l'adresse de retour et saut)\\
Appelé :\\
4 : sauvegarde le frame pointeur courant\\
5 : allocation d'espace sur la pile pour les variables locales\\
6 : code ...\\
7 : dépile des variables locales\\
8 : restaure le frame pointer\\
9 : appel à ret : dépile l'adresse de retour et jump\\
Appelant :\\
10 : dépile les paramètres de la fonctions\\
11 : restaure les registres\\

\subsection{Gestions des variables et tableaux}

Comme sous entendu précédemment, les arguments et variables locales scalaires sont stockées quelque part sur la pile. Leurs position est calculées
par rapport à \%ebx, le frame pointer.
Toutes les variables locales sont déclarées dans la section .data du binaire.
Les tableaux sont quant à eux alloués sur le tas. Chaque tableau est alloué au début du block (counpound statement) auquel il appartient et est
libéré à la fin de celui ci.\\
Pour des raisons de performances, l'instruction SSE movaps de déplacement de vecteur impose
que les adresses utilisées soit alignées sur (multiple de) 16 octets. Nous faisons pour celà appel à la fonction externe de la libc posix\_memalign
se chargeant de l'allocation.\\

Il n'y a dans notre code aucune gestion fine des registres. Optimiser leurs utilisations serait un des points majeur d'amélioration de notre
compilateur. La principale difficulté est de construire une structure de données afin d'abstraire au maximum la génération de code.

\subsection{Fonctionnalités implémentées}

Nous avons au cours de ce projet implémenté les fonctionnalités suivantes en assembleur :

Les opérations non mixtes suivantes ont été implémenté :\\
incrémentation (++), décrémentation ($--$), assignement (=), addition (+=), soustraction (-=), et multiplication (*=) ainsi que la division (/=)
pour les quatres types INT, FLOAT, INT[], FLOAT[] (même type à gauche et à droite de l'opérateur).\\

Pour tout ces types, il a été également implémenté l'opérateur print affichant la variable sur la sortie standard. Un appel à printf est effectué et 
la chaine de caractère utilisée pour le formatage est stockée dans la section data du binaire.\\

Pour le type INT, les instructions standard x86 (addl, subl, imull) ont été utilisées. Pour ce qui est des nombres flottants, nous avons dû nous
pencher sur la documentation du FPU (The Art of Assembly, chapter 13 et 14) (flds, fstps, faddp, fsubps, fmulp).
Comme nous l'avons dit, les instructions SSE pour être efficaces (movaps à la place de movups) nécessites des adresses mémoires alignées.\\

Deux instructions de conversion ont été écrit. (int = float) et (float = int) convertissent les entiers en flottants et réciproquement à l'aide
d'opérations effectuées par le FPU (filds et fists) (en omettant le choix du mode d'arrondi grâce au control register).\\

Nous avons comme demandé également implémenté quelques instructions mixte scalaire - vecteurs tel que :\\
(int += int vecteur, int *= int vecteur, float += float vecteur et float *= float vecteur) effectuant somme et produit de tout les éléments
d'un même vecteur. Notons par ailleurs que le produit scalaire entre deux vecteurs peut être décomposé en opérations deux adresses 
(mise au carré des éléments du vecteur puis somme).\\

L'opérateur sqrt float, calcul la racine carré d'un flottant.\\

Enfin, trois autres opérateur vectoriel ont été introduit et codé conformément au sujet : min=, max= et norm= calculant respectivement
le minimum, le maximum et la norme du vecteur.\\

La récursivité des fonctions ne semble pas poser pas de problème particulier.\\

Les opérations de conditions (==, <, >, <=, >=, !) n'ont été implémenté que sur les entiers mais peuvent relativement simplement être étendues
aux flottants à l'aide des instructions proposées par le FPU.\\

L'écriture de ce code a lui aussi été assez fastidieux et aurai été grandement allégé par une structure de donnée plus réfléchie.

\section{Difficultés rencontrés}
\begin{itemize}
\item Utilisation Lex/Yacc: leurs fonctionnement avancé n'est pas totalement indiqué lors des TDs.
\item Utilisation FPU et conversion entier - flottant: le fonctionnement du FPU, de ses registres formant une pile doit être compris.
\item Difficulté de déboguage: il s'est avéré que le débogage du binaire produit est long et pénible. Même avec des outils indispensables tel que gdb ou valgrind, cette opération reste difficile. Pire encore, nous avons été confronté à certains problèmes où gdb et valgrind ne possédait
pas le même résultat. (Vagrind à recoder des fonctions tel que malloc induisant des comportements ardu à démêler).
\item Manque de structuration du code
\end{itemize}

\section{Tests et Usage}

Voici les commandes incluent dans notre makefile, qui permettent de tester notre backend.
\begin{itemize}
\item Pour ne compiler que le code source "make".
\item Pour produire du code assembleur des tests invoquer "make test".
\item Pour assembler et lancer les tests invoquer "make run".
\item Pour lancer un petit test de performance invoquer "make perf".
\item Pour nettoyer invoquer "make clean".
\end{itemize}
\section{Conclusion}

Les opérations matricielles n'ont pas été traitées. Contrairement à gcc, nous n'avons pas géré l'alignement de la pile qui d'après internet
améliore considérablement les performances sur l'architecture que nous considérons. Bien que quelquez assert ont été posés, le backend 
suppose assez largement que le code d'entré est correcte.\\

Loin d'être simple, l'écriture d'un compilateur nécessite une forte connaissance technique de l'architecture des processeurs cible. De plus, 
également, nous nous somme rendu compte qu'un code efficace, clair et modulable demandait un travail très important du point de vue du
génie logiciel et de la structuration des données manipulé.

\end{document}

