\documentclass{article}

\title{Projet de Compilation, Backend}
\author{Lun Juang, Quentin Rouxel}
\date{23/01/2012}

\begin{document}
\maketitle{}

\section{Introduction}

L'objectif de ce projet était la réalisation d'un petit compilateur basé sur un langage proche du C et orienté vectoriel.
Nous nous sommes attaché à l'écriture du backend de ce compilateur. A partir d'un langage intermédiare dont la syntaxe et la sémantique
a préalablement été vérifié par le frontend, il s'agissait de générer le code assembleur correspondant et à destination d'une architecture
x86, 32 bits.

\section{Langage intermédiare}

\subsection{Réécriture de la grammaire}

Il s'est avéré que le fichier yacc implémentant la grammaire du langage intermédiare qui était fournie comportait quelque erreurs.
Plus exactement, la grammaire acceptait des codes qui ne faisait clairement pas partie ni du langage d'entrée, ni du langage intermédiaire.
Pour ne pas avoir a gérer des erreurs syntaxiques, nous avons choisi de réécrire en grande partie la grammaire 
pour d'une part, la rendre plus strict, plus cohérente avec la langage intermédiare utilisé ici et d'autre part un peu plus simple à
utilisé dans le cadre de l'arbre syntaxique.

\subsection{Utilisation de Lex et Yacc}

Nous nous somme basé sur ce document pour comprendre le fonctionnement détailler de Lex et de Yacc.
http://www.scribd.com/doc/8669780/Lex-yacc-Tutorial
L'instruction union de yacc permet de spécifier le type des valeurs contenue dans l'arbre syntaxique. Puis la déclaration
\%token<nom-type> (respectivement \%type<nom-type>) permet de définir le champ à utiliser dans le union pour atteindre la valeur des token, 
respectivement des règle de la grammaire.

\subsection{Structure de donnée}

Au lieu de générer le code assembleur directement dans les règles sémantique de Yacc, nous avons préféréé nous abstraire du parcours 
postfixe assez contraingnant de l'arbre syntaxique. ce parcours est imposé par l'algorithme réalisant n'analyse syntaxique. Nous avons
pour ce faire recréé la structure complète de arbre syntaxique afin de la manipuler plus simplement par la suite. Ce choix s'explique 
également par la nécessite de pouvoir manipuler le code si par la suite, nous voulions faire des optimisation dessus.
Le fichier ecc.h et ecc.c contiennent respectivement la définition et l'implémentation de cette structure de l'arbre syntaxique.
Les règles sémantiques associé à chaque règle de la grammaire ont alors pour role de construire la représentation du code et de le faire
remonter par l'arbre syntaxique.

Nottons ici que l'utilisation d'un langage orienté objet aurrai simplifier cette construction fastidieuse.

Deux opération importante sont effectué après la construction de la structures de données et avant la génération du code : 
le chainage des tables de variables et le calcul des offsets pour chaque variable.
Il s'agit pour le premier de lier se qui nous sert de table de symbole pour les variables à la table englobante.
Pour le deuxième, on calcul par rapport à \%epb, le décalage à effectuer sur la pile pour accéder à la variable donnnée. Positif 
pour les arguments d'une fonction, négatif pour les variables locales.

\section{Génération du code assembleur}

\subsection{Assembleur}
\subsection{Gestions des variables et tableaux}

Comme sous entendu précédement, toutes les variables, arguments et locales sont stocké quelque part sur la pile.
\subsection{Opérations implémentées}



\section{Difficultés rencontrés}
-> utilisation Lex/Yacc
-> utilisation fpu, doc, conversion entier floatant
-> SSE : alignement et padding
-> gestion de la pile
-> cas particulier, tableaux global
-> difficile à débugé
-> manque de structuration du code -> meilleir réfléxion nécessaire (modularité difficile avec machine à état) -> structure nécessaire

\section{Conclusion}

\end{document}

\begin{comment}
-> grammaire : en partie réécriture
-> utilisation de Lex et Yac (struct donnée)
-> Structure de données : enregistrement intégral de l'arbre syntaxique
-> Structure du code, génération de l'asm après 

-> Point technique : chainage table variable, calcul des offsets, ...

-> Non gestion des erreurs (réécriture grammaire)

-> Génération de l'assembleur
-> protocole d'appel
-> gestion de la pile
-> Allocation/désaloccation
-> utilisation du fpu
-> utilisation de sse
-> Opération implémentée

-> Limitation : pas d'optimisation de la gestion des registres (ou non traitées)
-> pas d'alignement de la pile
-> op non implémenté

-> Test et usage

\end{comment}

